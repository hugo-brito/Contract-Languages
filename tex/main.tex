\documentclass{ituthesis}

\settitle{Domain-Specific Languages\\for Analysing and Testing\\Properties of Contracts}
\setauthor{Hugo Brito \\ \href{mailto:hubr@itu.dk}{hubr@itu.dk} \\ \\ Jonas Andersen \\ \href{mailto:haja@itu.dk}{haja@itu.dk}}
\setsupervisor{Willard Rafnsson}
\setextrasupervisor{Patrick Bahr}
\setdate{December 2019}

\begin{document}
% Uncomment the following to set the default language of the document to Danish. This affects hyphenation as well as headers and the like.
%\selectlanguage{danish}

\frontmatter

\thetitlepage
\newpage

\begin{abstract}
This is an English abstract. English is the default language of this document.
\end{abstract}

\begin{otherlanguage}{danish}
  \begin{abstract}
    Dette er et resumé på dansk, som er det alternative sprog i dokumentet.
  \end{abstract}
\end{otherlanguage}

\cleardoublepage
\setcounter{tocdepth}{1}
\tableofcontents

\mainmatter

%from memoir documentation:
%TeX tries very hard to keep text lines justified while keeping the interword spacing as constant as possible, but sometimes fails and complains about an overfull hbox.
%The default mode for LaTeX typesetting is \fussy where the (variation of) interword spacing in justified text is kept to a minimum. Following the \sloppy declaration there may be a much looser setting of justified text.
%Additionally the class provides the \midsloppy declaration which allows a setting somewhere between \fussy and \sloppy.
%fewer overfull lines than \fussy, and fewer obvious large interword spacing than with \sloppy.
%the memoir manual also uses \midsloppy!
\midsloppy
% try harder to avoid widows and orphans
\sloppybottom

\chapter{Introduction}
	- 15 pg. + 2 pg per group member (19 pg max)
	- Go for around 17 pages. Should be self-contained, someone who was not in these meetings should be able to understand the report. 
	- Report should be lean and conrete, it is okay if it takes up less pages.
	- If unsure about omitting or including stuff in report, can consult them.
We should start working on sections, so we can get feedback from them as much as possible.

-------------------------


	- Explain what a contract language is
	- Explain what's out there, in terms of Research
	- What are Contract Languages used for?
		○ Use an example as motivation to tell a story
		○ Use an example to show how is used and how implemented (doesn't have to be own)
		○ tell a story, we want to describe something and why ours looks different (because it explains what it is supposed to express)
	- Show our contract (show the actual contract and explain with code examples how it works)
		○ What is our intention of doing it like so?
		○ Define our language (Evalulation function should do this)
	- Pick some properties that could be interesting to consider for our language, but do not implement property based testing.
A well written F# (Scala) example should be fine, not necessary to formally define in logic.

\originally{History of the Makhnovist Movement; 1923}
\joint{Peter Arshinov}
\chapter{Something in Between}
Somewhere at sometime...

\section{A Section}
\subsection{A Subsection}
Some text.
\subsubsection{A subsubsection}
More text here.

Here's some verbatim text:
\begin{verbatim}
data Vect : Nat -> Type -> Type where
  Nil : Vect 0 a
  (::) : a -> Vect n a -> Vect (1 + n) a
\end{verbatim}

\joint{Emma Goldman}
\chapter{At the Same Time Somewhere Else}
... happened something else

Here's some mathematics:
\begin{displaymath}
  \frac{
    \Gamma, x:\tau_1\;\vdash\; e \;:\; \tau_2
  }{
    \Gamma\;\vdash\;\lambda x : \tau_1 . e \;:\; \tau_1\to\tau_2}
\end{displaymath}

\originally{My autobiography; 1934}
\chapter{Conclusion}
Finished


\end{document}
