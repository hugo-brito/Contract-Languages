\documentclass{ituthesis}

\settitle{Domain-Specific Languages\\for Analysing and Testing\\Properties of Contracts}
\setauthor{Hugo Brito \\ \href{mailto:hubr@itu.dk}{hubr@itu.dk} \\ \\ Jonas Andersen \\ \href{mailto:haja@itu.dk}{haja@itu.dk}}
\setsupervisor{Willard Rafnsson}
\setextrasupervisor{Patrick Bahr}
\setdate{December 2019}

\begin{document}
% Uncomment the following to set the default language of the document to Danish. This affects hyphenation as well as headers and the like.
%\selectlanguage{danish}

\frontmatter

\thetitlepage
\newpage

\begin{abstract}
We develop, present and explain the work developed around a declarative language for the formalisation of and analysis of contracts. The present has taken inspiration from the available literature on the matter.
In this paper, starting from a simple example contract, we implement a very simple Contract Specification Language\footnote{A formal language for contracts: \url{https://www.deondigital.com/2019/03/introducing-csl/}}, test it and reason about it. The goal is to provide the reader with an introductory view on the topic, as well as laying the foundations for the next step of this process, which will be to use Property-Based Testing\footnote{What is Property Based Testing?:\\ \url{https://hypothesis.works/articles/what-is-property-based-testing/}} to improve the necessary computations as well as on contracts built with this tool.
\end{abstract}

\begin{otherlanguage}{danish}
  \begin{abstract}
    Dette er et resumé på dansk, som er det alternative sprog i dokumentet.
  \end{abstract}
\end{otherlanguage}

\cleardoublepage
\setcounter{tocdepth}{1}
\tableofcontents

\mainmatter

%from memoir documentation:
%TeX tries very hard to keep text lines justified while keeping the interword spacing as constant as possible, but sometimes fails and complains about an overfull hbox.
%The default mode for LaTeX typesetting is \fussy where the (variation of) interword spacing in justified text is kept to a minimum. Following the \sloppy declaration there may be a much looser setting of justified text.
%Additionally the class provides the \midsloppy declaration which allows a setting somewhere between \fussy and \sloppy.
%fewer overfull lines than \fussy, and fewer obvious large interword spacing than with \sloppy.
%the memoir manual also uses \midsloppy!
\midsloppy
% try harder to avoid widows and orphans
\sloppybottom

\chapter{Introduction}

%% Possible commands and headers:
% (Copied from the original template)

% \originally{History of the Makhnovist Movement; 1923}
% \joint{Peter Arshinov}
% \chapter{Something in Between}
% Somewhere at sometime...

% \section{A Section}
% \subsection{A Subsection}
% Some text.
% \subsubsection{A subsubsection}
% More text here.

% Here's some verbatim text:
% \begin{verbatim}
% data Vect : Nat -> Type -> Type where
%   Nil : Vect 0 a
%   (::) : a -> Vect n a -> Vect (1 + n) a
% \end{verbatim}

% \joint{Emma Goldman}
% \chapter{At the Same Time Somewhere Else}
% ... happened something else

% Here's some mathematics:
% \begin{displaymath}
%   \frac{
%     \Gamma, x:\tau_1\;\vdash\; e \;:\; \tau_2
%   }{
%     \Gamma\;\vdash\;\lambda x : \tau_1 . e \;:\; \tau_1\to\tau_2}
% \end{displaymath}

% \originally{My autobiography; 1934}
% \chapter{Conclusion}
% Finished

% Report guidelines:
% \begin{itemize}
%     \item 15 pg. + 2 pg per group member (19 pg max)
%     \item Go for around 17 pages. Should be self-contained, someone who was not in these meetings should be able to understand the report. 
%     \item Report should be lean and concrete, it is okay if it takes up less pages.
%     \item If unsure about omitting or including stuff in report, can consult them.
% \end{itemize}
% We should start working on sections, so we can get feedback from them as much as possible.

% -------------------------

% \begin{itemize}
%     \item Explain what a contract language is
%     \item Explain what's out there, in terms of Research
%     \item What are Contract Languages used for?
%     \begin{itemize}
%         \item Use an example as motivation to tell a story
%         \item Use an example to show how is used and how implemented (doesn't have to be own)
%         \item tell a story, we want to describe something and why ours looks different (because it explains what it is supposed to express)
%     \end{itemize}
% 	\item Show our contract (show the actual contract and explain with code examples how it works)
% 	\begin{itemize}
% 	    \item What is our intention of doing it like so?
% 	    \item Define our language (Evaluation function should do this)
% 	\end{itemize}
% 	\item Pick some properties that could be interesting to consider for our language, but do not implement property based testing.
% \end{itemize}
%
% A well written F\# (Scala) example should be fine, not necessary to formally define in logic.
In this report we will investigate how to develop our own contract language in the domain of a simple exchange of resources. We are doing this to lay the foundation for our thesis, which will be on Property-Based Testing on contracts formalised using a contract language.

We will start by giving a brief introduction to the core concepts of this paper and the problem we are trying to solve. The concept of a Contract Language will be more thoroughly explored in the Methodology section.

% \item Explain what a contract language is
\section{What is a Domain-Specific Language?}
% Does it answer the question: What are Contract Languages used for?

% very good explanation on what a https://www.jetbrains.com/mps/concepts/domain-specific-languages/
A \textit{domain-specific language} (from now on DSL) is a programming language tailored to a specific problem domain. Contrary to what happens with a general-purpose programming language, which use is not restricted to a particular type of problem or machine, a DSL is built with the sole purpose of modelling and expressing a specific set of problems \cite{van2000domain}.

By default, it is less complex to use than a General-Purpose Programming Language and it is meant to be developed in close contact with domain experts, as ultimately these will constitute its users \cite{dsl}.

\section{What is a Contract Language?}
A contract language is a \textit{domain-specific language} for contracts. Unlike most other DSLs the purpose of a contract language is to make contracts that can be treated like data, hence it cam be executed several times. Besides this, these contracts are subjected to monitoring and various user defined analyses \cite{andersen2006compositional}.

Unfortunately, not all contracts can be implemented using a contract language as it requires the given contract to be unambiguous in nature, such as financial contracts and in general contracts governing the exchange of resources.

% but it is stricter in the sense that it is part of smaller subset of the DSLs that are built and used only in the context of Contracts. Specifically contracts that tend to be unambiguous in nature, hence Financial Contract, Exchange of Resources etc.
% This kind of contracts allows for an extreme simplification of the DSL since it does not deal with, e.g. perishable goods, etc. By the fact they deal mainly with amounts of money, which does not require, e.g. somewhere to be stored, does not perish, can be easily transferred (does not require transportation), they are simpler to model, reason and work with.

\section{What is Property-Based Testing?}
Testing is an important step when making any piece of software available to the general public. It helps saving money, enforces safety, and improves product quality and customer satisfaction. % source: https://www.testdevlab.com/blog/2018/07/importance-of-software-testing/
Classical techniques to achieve this purpose include white and black box testing. % source: https://usersnap.com/blog/software-testing-basics/
In order to understand the advantages and major differences between the most common ways of testing software and property-based testing, it is important to mention the principles in which the previous operate.
\begin{itemize}
    \item Black-box testing takes a piece of software as a black box: it inputs some data and knows beforehand what should the output be. It does not need to know the source code as it does not care how the software achieves the result. It focuses on functionality. % source: https://usersnap.com/blog/software-testing-basics/
    \item White-box testing requires access to the source code and tests the internal structures of a software. It checks for branches, statement and decision coverage, among others. The goal is to understand which line of the code is being executed and identify how the execution of the software can branch and what the correct output should be. % source: Williams, Laurie. "White-Box Testing" (PDF): 60–61, 69.
    % https://students.cs.byu.edu/~cs340ta/spring2019/readings/WhiteBox.pdf
\end{itemize}

Property-Based Testing (from now on PBT) takes testing to another level. It can be understood as a combination of white and black-box testing, but instead of testing either on the entirety of an application or some subset of it, it does both. PBT relies on properties (instead of concrete inputs) to test correctness. It checks that the function, program or in our case a contract abides by the given property. Properties are "rules" that the output of given program, function or contract should always follow \cite{fink1994towards}.

In this project we are not going to explore PBT on contracts in detail, as this is the focus of our thesis. We will, however, try to formulate some properties of the contracts that could then be later tested on.

\section{Problem Formulation}
The point of this report is for us to gain an understanding of how to implement and use a DSL to describe contracts. With that in mind we aim to do the following:

\begin{itemize}
    \item Gain insight to the literature in this area along with what kind of contracts one can implement.
    \item Create a working prototype of our contract language.
    \item Be able to do simple analysis on the language
    \item Find some properties that should hold true for the language.
\end{itemize}

Adding to this we are also trying to figure out the benefit of contract languages.

% Starting from a contract described in natural language:
% \begin{itemize}
%     \item Do any parts of the contract promise too much?
%     \item What are the possible ways that this contract can be run?
%     \item What happens if a contract is delivered only partially?% execution of contracts ( and use it to renegotiate a contract of what could follow (rights, money, goods)
%     %\item What would happen if there are mismatches in volumes?
%     %\item How much will it cost the logistics service provider if direct damages or indirect damages occur on the goods?
%     %\item What if the logistics service provider cannot deliver at the right quality (in terms of Key Performance Indicators)? How much is the shipper allowed to withhold of the payment? What if expectations are exceeded? How much is there to be gained?
%     %\item Which events will lead to early termination of the contract? What effect will this have on the logistics service provider?
% \end{itemize}

% *Probably need to change this part:*    
% A potential application scenario could be within logistics, where one considers in which circumstances a logistics service provider is able to withhold goods from the recipient. Moreover, if such was to happen, how much cost would be covered by selling the goods?

\chapter{Methodology}
% Does it answer the question: Explain what's out there, in terms of Research
% *One of the keypoints of the meetings:\\
% Explain what's out there, in terms of Research*

% Explain the approach to our paper, how did we make the language and how did we get there.

In this paper we use the inductive approach to determine how to develop our Contract Language. This section will explore what research we have been through to give an understanding of our process and why our implementation looks the way it does.

\section{Background}
DSLs have been around almost as long as the the existence of humans, as we always find a way to simplify language within certain domains. Musicians use musical notation, lawyers use legal-language and we programmers use code. Nowadays most programmers use general purpose languages that might contain DSLs to make the solving of certain problems easier, whereas older languages such as Cobol were made as dedicated languages to solve a certain range of problems \cite{van2000domain}.

However, the idea of using a DSL to give a formal description to financial contracts in code, where not introduced until the 2000's by Peyton-Jones, Eber \& Seward. They specifically explored how to express Financial Contracts, such as bonds, stocks etc. and even made a combinator library \cite{peyton2000composing}. This type of DSL is known as a Contract Language. To this day Contract Languages are still most commonly used in the domain of the Financial industry. One of the first companies to utilise this was LexiFi which has since been followed by most banks and other companies such as SimCorp.

---- Other Research papers read to be included ----

There have been made a general contract language that can encompass more than just financial contracts, by extending the language of Peyton-Jones, Eber \& Seward. This language formalises contracts governing the exchange of resources, hence it encompasses the exchange of money, goods and services among multiple parties \cite{andersen2006compositional}. The aforementioned paper laid the groundwork for the contract language of developed by Deon Digital, which has provided us with valuable guidance in how to make our own contract language. Our implementation of the contract language itself will be following the ideas from this paper quite closely.

\section{Why a DSL for Contracts?}
% Motivation section
% Use an example to show how is used and how implemented (doesn't have to be own) 
% Based on an example: *Why a DSL*
% what is it good for in this case?

This part will elaborate on the benefits of making a contract as a DSL. After it will use an example from a paper to show how it is used and implemented.



\begin{tcolorbox}
\textbf{Section 1.} The attorney shall provide, on a non-exclusive basis, legal services up to ($n$) hours per month, and furthermore provide services in excess of ($n$) hours upon agreement.\par
\textbf{Section 2.} In consideration hereof, the company shall pay a monthly fee of (amount in dollars) before the $8^{th}$ day of the following month and ($rate$) per hour for any services in excess of ($n$) hours 40 days after the receival of an invoice.\par
\textbf{Section 3.} This contract is valid from the $1^{st}$ of January until the $31^{st}$ of December $2020$.
\end{tcolorbox}


\begin{tcolorbox}
\begin{itemize}
    \item Alice wishes to buy a bike from Bob.
    \item Each bike Bob has for sale costs any buyer 100 €
    \item 
\end{itemize}
\end{tcolorbox}

% Nowadays many banks and financial institutions make use of their implementation of a Contract Language as it enables systematic the analysis of their financial contracts. However, while this is the norm in the financial industry, other industries have yet to realise the potential of Contract Languages.

\chapter{Implementation}
This chapter is devoted to the code.\par
Start with a concrete contract in plain English and build code along as we need.\par
In here we should have snippets of code, annotated with comments that connect the dots between our example contract and the code.\par
Depending on our example contract, we might also mention some concepts (that are not necessarily code-related, but related to the domain of the contract) that the reader needs to know or be aware of so that the implementation becomes clearer.\par
Some questions that we would like to have answered here:
\begin{itemize}
    \item Why does the code look the way it does?
    \item What models what and how in our code?
    \item What isn't modelled and why? (limitations --- basically, what can't be expressed with our contract language)
\end{itemize}


\chapter{Testing}
This chapter will be included if we have time for it.
This chapter is devoted to assessing the correctness of our code.\par
We plan on having a test suite for our language, but this part of the project is still under development (the actual language comes first).\par
Things to consider:
\begin{itemize}
    \item Language correctness
    \item Logic inference rules when running the contract (?)
\end{itemize}


\chapter{Conclusion}
In this chapter we draw conclusions about the language, the works developed and we look in to the future: the thesis.



\bibliographystyle{apalike}
\bibliography{references}
\printbibliography


\end{document}
